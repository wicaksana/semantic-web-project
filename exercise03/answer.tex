\documentclass[pape=a4, fontsize=12pt]{article}
\usepackage{amsmath}

%%%%%%%%%%%%%%%%%%%%%%%%%%%%%%%%%%%%%%%%%%%%%%%%%%%%%%%%%%
% Title
%%%%%%%%%%%%%%%%%%%%%%%%%%%%%%%%%%%%%%%%%%%%%%%%%%%%%%%%%%
\author{Group 2/04}
\title{\textbf{Answer of Question 2.8.a}}
\date{}

%%%%%%%%%%%%%%%%%%%%%%%%%%%%%%%%%%%%%%%%%%%%%%%%%%%%%%%%%%
% Start the document
%%%%%%%%%%%%%%%%%%%%%%%%%%%%%%%%%%%%%%%%%%%%%%%%%%%%%%%%%%
\begin{document}
\maketitle
%% see the following for the math symbol list
%% http://web.ift.uib.no/Teori/KURS/WRK/TeX/symALL.html
\noindent
Using the assumption of ontology description mentioned in question 2.6.b: \\

\noindent
Sibling $ \equiv $ Individual $ \sqcap$ $ \exists$childOf.(Family $ \sqcap>$1 parentOf.Individual) \\
Brother $ \equiv $ Sibling $ \sqcap $ Male \\
Sister $ \equiv $ Sibling $ \sqcap $ Female \\
%%Uncle $ \equiv $ Male $ \sqcap $ $ \exists$siblingOf.Parent \\
Uncle $ \equiv $ Brother $ \sqcap $ $ \exists$childOf.Sibling \\
Aunt $ \equiv $ Sister $ \sqcap  $ $ \exists$childOf.Sibling \\
Grandparent $ \equiv $ Individual $ \sqcap $ $ \exists$parentOf.Parent \\
Grandfather $ \equiv $ Grandparent $ \sqcap $ Male \\
Grandmother $ \equiv $ Grandparent $ \sqcap $ Female \\
Grandchild $ \equiv $ $ Grandparent^{-} $ \\
Grandson $ \equiv $ Grandchild $ \sqcap $ Male \\
Granddaughter $ \equiv $ Grandchild $ \sqcap $ Female \\
\end{document}